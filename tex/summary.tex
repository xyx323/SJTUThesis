%# -*- coding: utf-8-unix -*-
% !TEX program = xelatex
% !TEX root = ../thesis.tex
% !TEX encoding = UTF-8 Unicode
%%==================================================
%% conclusion.tex for SJTUThesis
%% Encoding: UTF-8
%%==================================================
\chapter{总结与展望}
\label{chap:conculution}
\section{本文贡献}

目前,数据库技术被广泛地应用于软件系统作为数据存储的主要方式。
除了软件开发人员需要编程进行数据存取外,业务人员在OLAP和报表等应用场景中也需要编写SQL查询语句,从而实现灵活的数据按需读取。
然而,掌握SQL语言需要使用者经过数据库和SQL相关知识的培训,且在使用时需要确切了解数据库的模式等信息。
如果可以让使用者通过交互式的自然语言接口甚至直接使用自然语言来对数据库执行查询将大大缩短业务与技术的鸿沟,极大地提升软件的生产效率。
在此背景下,本文对面向最终用户的SQL查询语句生成技术进行了研究,提出了基于映射的INL2SQL生成、基于深度强化学习的NL2SQL生成和基于多任务学习的NL2SQL生成三种方法。

本文的主要贡献和创新点包括:

\textbf{1)基于映射的交互式自然语言接口生成SQL查询语句的方法}

本文提出了基于映射的交互式自然语言接口生成SQL语句的模型与方法,其核心思想是结合自然语言解析技术,使用人机交互的方式指导解析过程,
通过依赖解析树生成、解析树节点映射、解析树优化重构、查询树翻译模块对用户输入的查询进行意图的解析并映射到SQL查询语句上。
从Classicmodels和MAS数据集上的实验结果来看,模型在有交互的情况下在简单、中等、困难的场景下的表现为100\%、80\%、35\%和100\%、93\%、71\%的准确性。
通过交互式对话器和用户接口模块对意图解析和映射进行补充和重构,减少了自然语言解析中出现的歧义问题和错误情况,提升了SQL查询语句生成的有效性。


\textbf{2)基于深度强化学习的英文自然语言生成SQL查询语句的方法}

本文提出了一种基于深度强化学习的NL2SQL生成的模型与方法。
本方法在离线训练阶段,将WikiSQL中的数据进行序列表示后输入给网络模型,模型生成SQL查询语句后在数据库中执行,再使用奖励函数构造奖励用于强化网络模型。
在在线生成阶段用户输入自然语言查询语句及数据库模式,训练好的网络模型将生成符合用户查询意图的SQL查询语句。
本方法的网络模型,即增强解析器模型,由编码器和解码器构成,并结合了自注意力机制。
通过引入强化学习技术,本方法将增强解析器学习的目标转换为策略的优化问题并对给出了解析器的状态和动作的定义。
为了解决SQL查询语句中的过滤条件的顺序问题以及隐式列名的问题,还提出了非确定性预言和$ANYCOL$状态的解决办法。
实验表明,与目前NL2SQL的现有方法相比,本方法在WikiSQL数据集上的表现一流,并且在Spider数据集的验证集和测试集上的表现超过同类方法,其逻辑形式准确率达到23.2\%和24.1\%。


\textbf{3)中文自然语言生成SQL查询语句的方法}

为了进一步提高NL2SQL生成的准确性以及解决中文自然语言生成SQL查询语句的问题,本文提出了一种基于多任务学习的NL2SQL生成的模型与方法。
本方法使用TCR模板把中文自然语言转换为英文自然语言任务和英文自然语言转换为SQL查询语句任务进行统一。
再使用由编码器和解码器构成的多任务网络模型进行多任务学习。
其中,编码过程包含单独编码、对准、对偶协同注意力、压缩、自注意力和最终编码阶段;
解码过程包含自注意力、获得中间状态、任务与内容注意力、多指针生成阶段。
本文进行了多任务学习实验和优化策略实验,在WikiSQL数据集上的逻辑形式准确率和数据库执行准确率达到78.7\%和86.1\%的最高水平,并且在引入更多的任务进行学习时达到了607.7的总分。
实验表明,本方法不仅能够提升英文自然语言生成SQL查询语句任务的准确性,还能够很好地完成中文自然语言生成SQL语言任务,同时还具有很好的可拓展性。


\section{展望}

由于时间和能力有限,本文的研究工作还有不少不足,我们计划在以下两个方面做进一步的研究和改进:

\textbf{1)交互式自然语言接口生成SQL查询语句的泛化能力}

从本文的实验结果可以看出,模型存在无法处理子查询、所需输入的自然语言意图形式较固定、无法处理特殊词组等问题,需要进一步进行提升和改进。
可以考虑使用深度学习进行语义分类、命名实体识别等手段辅助进行意图的提取。
还可以采用将深度学习和模块化的工程相结合的方法,提升系统对于子查询的处理能力。
除此之外,本文的交互方法较为简单,接下来还可以考虑将任务型问答系统的相关技术(例如,槽填充技术)运用在与用户的多轮交互中,从而更好地解决歧义和信息缺失的问题。


\textbf{2)复杂SQL语句的自动生成}

本文及其他最新的研究工作所使用到的数据集都是以WikiSQL数据集为主,而WikiSQL数据属于Spider数据集中定义的简单和中等难度。
本文提出的解决方案已经几乎达到简单和中等难度的SQl查询语句生成的最佳性能,之后的模型需要更多地考虑解决语句的多样性和复杂性。
可以通过分析困难和极难难度的SQL查询语句的特性来设计新的更复杂的神经网络结构,可以考虑将深度学习和模块化的工程相结合的办法解决单个神经网络无法解决的问题。
另外,SQL查询语句只是SQL语句中结构最简单、语义最清楚的一种,而范围更大的SQL语句的自动生成任务是更具挑战的。

